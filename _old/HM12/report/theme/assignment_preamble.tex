% --------------------------------------------------------------------
% Packages
% --------------------------------------------------------------------
\usepackage[a4paper, margin=1.5cm]{geometry}

% Text, input, language-related packages
\usepackage[utf8]{inputenc}
\usepackage[T1]{fontenc}
\usepackage[english]{babel}
\usepackage[protrusion=true,expansion=true]{microtype} 

% Math packages
\usepackage{amsmath,amsfonts,amsthm,amssymb,bm,mathdots,mathtools,bigints}

% TiKz packages
\usepackage{tikz, pgfplots, tikzsymbols, tikzducks}

% Graphic and color packages
\usepackage[export]{adjustbox}				% Environment to adjust LaTeX objects
\usepackage[skins]{tcolorbox}				% Coloured boxes for LaTeX objects
\usepackage{graphicx} 						% Enhanced support for graphics
\usepackage{color, xcolor}					% Driver-independent color extensions
\usepackage{wrapfig}						% Environment to wrap figures/tables
\usepackage{subcaption}						% Environment to create subcaptions

% General utilities
\usepackage[nodayofweek]{datetime}			% Customize date commands
\usepackage[printwatermark]{xwatermark}		% Watermarks the document
\usepackage[document]{ragged2e}				% Text-alignment (\centering, ...)
\usepackage{stfloats}						% Control presentation of floats
\usepackage{framed, mdframed}				% Framed and shaded environments
\usepackage{multicol}						% Multicolumn environment
\usepackage{environ}						% Better interface for environments
\usepackage{minted, listings}				% Coding environments
\usepackage{hyperref}				        % Support for hypertext and links
\usepackage{enumerate}				        % Options for enumerate environments
\usepackage{etoolbox}						% Environment hooks (\BeforeBegin...)
\usepackage{parskip}						% Manages paragraph spacing stuff
% --------------------------------------------------------------------
% Packages Configurations
% --------------------------------------------------------------------
% (general): Define general paragraphing spaces and mono font
\setlength{\parskip}{0cm}
\setlength{\parindent}{0cm}

% (tikz): Imports some custom libraries and pgfplot settings
\usetikzlibrary{shapes, arrows, decorations.pathmorphing}
\usetikzlibrary{backgrounds, positioning, fit, calc}
\usetikzlibrary{bayesnet, graphs}

\pgfplotsset{compat=1.16} 

% (minted): Set styling configurations
\usemintedstyle{pastie}
\setminted{ linenos=true,             % Line numbers
            autogobble=true,          % Automatically remove common white space
            frame=lines,
            framesep=2mm}

\renewcommand{\theFancyVerbLine}{\sffamily {\normalsize \oldstylenums{\arabic{FancyVerbLine}}}}

% (listings): Defines the code styling and formatting
\lstset{
    backgroundcolor=\color[RGB]{39,40,34},
    language=matlab, keywordstyle=\color[RGB]{102,217,239},
    basicstyle=\footnotesize \ttfamily,breaklines=true,
    escapeinside={\%*}{*)}
}

% (datetime): Personalized date commands (apply them before a \today)
\newdateformat{datenum}{\twodigit{\THEDAY}/\twodigit{\THEMONTH}/\THEYEAR}
\newdateformat{datefull}{\twodigit{\THEDAY}{ }\monthname[\THEMONTH], \THEYEAR}

% (hyperref): Setup link colors and other properties
\hypersetup{
    colorlinks=true,
    linkcolor=black!95!white,
    filecolor=magenta,      
    urlcolor=cyan,
}


% --------------------------------------------------------------------
% Other Definitions
% --------------------------------------------------------------------
% Additional cool colors
\definecolor{stateColor}{cmyk}{0.82,0,0.92,0}
\definecolor{inputColor}{RGB}{73,158,255}
\definecolor{disturbanceColor}{RGB}{102,43,153}
\definecolor{outputColor}{RGB}{224,60,48}

% --------------------------------------------------------------------
% Commands
% --------------------------------------------------------------------

% A light title-page header
\newcommand*{\thetitle}{%
    \vskip0.5cm%
    \begin{minipage}{0.3\textwidth} % Left side of title section
        \raggedright
        \institute\\\course\\[0.5ex]\footnotesize\professor
        \medskip\hrule
    \end{minipage}\hfill
    \begin{minipage}{0.4\textwidth} % Center of title section
        \centering 
        {\huge \bfseries Homework \assnumber}\\[1ex]{\LARGE\title}\\
    \end{minipage}\hfill
    \begin{minipage}{0.3\textwidth} % Right side of title section
        \raggedleft
        {\author$\ $(\authorid)}\\\texttt{(\email)}\\[0.5ex]\footnotesize\today
        \medskip\hrule
    \end{minipage}
    \vskip1cm 
}

% Colored hyperlinks
\newcommand{\chref}[2]{
  \href{#1}{{\usebeamercolor[bg]{Feather}#2}}
}

% A cooler transpose symbol 'T'
\newcommand*{\tran}{{\mkern-1.5mu\mathsf{T}}}

% Cool tensor sub- and super-scripts (\tend and \tenq, respectively)
\newcommand{\tend}[1]{\hbox{\oalign{$\bm{#1}$\crcr\hidewidth$\scriptscriptstyle\bm{\sim}$\hidewidth}}}
\newcommand{\tenq}[1]{\hbox{\oalign{$\bm{#1}$\crcr\hidewidth$\scriptscriptstyle\bm{\sim}$\hidewidth}}}

% Argmax and Argmin math operators
\DeclareMathOperator*{\argmax}{arg\,max}
\DeclareMathOperator*{\argmin}{arg\,min}

% --------------------------------------------------------------------
% Environments
% --------------------------------------------------------------------

% Boxed-enviroment with counter for each problem
\newcounter{problem}
\makeatletter
\newenvironment{problem}
    {%
    \refstepcounter{problem} \begin{tcolorbox}[
    fuzzy shadow={0.2mm}{-0.2mm}{0mm}{0.1mm}{black!50!white},
    enhanced, shrink tight, 
    extrude by=3mm, 
    colframe=black,
    colback=black!5!white,
    arc=1pt,
    boxrule=0.5pt]
    {\bf Problem~\theproblem.}%
    }%
    {%
    \end{tcolorbox}\vskip0.5cm
    }

% Simple solution environment with end square
\makeatletter
\newenvironment{solution}
    {%
    {\bf Solution.}%
    }%
    {%
    \hfill$\square$
    }%

% Wrapper for the minted environment
\makeatletter
\newenvironment{code}[3][0.8\textwidth]
    {%
    \VerbatimEnvironment
    \begin{minipage}{\textwidth} \centering%
    \begin{minipage}{#1}%
    \begin{listing}[H]%
    \caption{#2}\label{#3}%
    \begin{minted}{julia}%
    }%
    {%
    \end{minted}
    \end{listing}%
    \end{minipage}%
    \end{minipage}%
    \vskip0.5cm
    }

% --------------------------------------------------------------------
% Unicode Characters Definition (for Julia Minted)
% --------------------------------------------------------------------
\DeclareUnicodeCharacter{3B4}{$\bm{\delta}$}
\DeclareUnicodeCharacter{3B8}{$\bm{\theta}$}
\DeclareUnicodeCharacter{3BB}{$\lambda$}
\DeclareUnicodeCharacter{3BC}{$\mu$}
\DeclareUnicodeCharacter{3BC}{$\pi$}
\DeclareUnicodeCharacter{3C3}{$\sigma$}
\DeclareUnicodeCharacter{3C0}{$\pi$}
\DeclareUnicodeCharacter{3D5}{$\bm{\phi}$}
\DeclareUnicodeCharacter{2208}{$\in$}
\DeclareUnicodeCharacter{2297}{$\otimes$}
\DeclareUnicodeCharacter{2A02}{$\bigotimes$}
\DeclareUnicodeCharacter{2299}{$\odot$}
\DeclareUnicodeCharacter{22C4}{$\diamond$}
\DeclareUnicodeCharacter{209A}{$_{p}$}
\DeclareUnicodeCharacter{221A}{$\sqrt{}$}
\DeclareUnicodeCharacter{2248}{$\approx$}
\DeclareUnicodeCharacter{2080}{$_o$}
\DeclareUnicodeCharacter{2090}{$_a$}
\DeclareUnicodeCharacter{03B1}{$\alpha$}
\DeclareUnicodeCharacter{1D65}{$_v$}
\DeclareUnicodeCharacter{2096}{$_k$}
\DeclareUnicodeCharacter{1D63}{$_r$}
\DeclareUnicodeCharacter{2081}{$_1$}
\DeclareUnicodeCharacter{2082}{$_2$}
\DeclareUnicodeCharacter{2083}{$_3$}
\DeclareUnicodeCharacter{2084}{$_4$}
\DeclareUnicodeCharacter{208C}{$_=$}
\DeclareUnicodeCharacter{208B}{$_-$}
\DeclareUnicodeCharacter{208A}{$_+$}
\DeclareUnicodeCharacter{2211}{$\Sigma$}
\DeclareUnicodeCharacter{03A3}{$\Sigma$}
\DeclareUnicodeCharacter{03A0}{$\Pi$}
\DeclareUnicodeCharacter{1D4E7}{$\mathcal{X}$}
\DeclareUnicodeCharacter{2093}{$_x$}
\DeclareUnicodeCharacter{1D67}{$_y$}
\DeclareUnicodeCharacter{2260}{$\neq$}
\DeclareUnicodeCharacter{208D}{$_($}
\DeclareUnicodeCharacter{2099}{$_n$}
\DeclareUnicodeCharacter{2C7C}{$_i$}
\DeclareUnicodeCharacter{209C}{$_t$}
\DeclareUnicodeCharacter{208E}{$_)$}
\DeclareUnicodeCharacter{1D4E8}{$\mathcal{Y}$}
\DeclareUnicodeCharacter{2A09}{$\times$}
\DeclareUnicodeCharacter{207D}{$^($}
\DeclareUnicodeCharacter{2074}{$^4$}
\DeclareUnicodeCharacter{2075}{$^5$}
\DeclareUnicodeCharacter{207F}{$^n$}
\DeclareUnicodeCharacter{1D48}{$^d$}
\DeclareUnicodeCharacter{1D3A}{$^N$}
\DeclareUnicodeCharacter{1D34}{$^H$}
\DeclareUnicodeCharacter{207E}{$^)$}
\DeclareUnicodeCharacter{207B}{$^-$}
\DeclareUnicodeCharacter{207A}{$^+$}
\DeclareUnicodeCharacter{1D4E2}{$\mathcal{S}$}
\DeclareUnicodeCharacter{1D4E4}{$\mathcal{U}$}
\DeclareUnicodeCharacter{1D40}{$^T$}
\DeclareUnicodeCharacter{03F5}{$\epsilon$}
\DeclareUnicodeCharacter{022EF}{$\cdots$}
\DeclareUnicodeCharacter{1D62}{$_i$}
\DeclareUnicodeCharacter{2218}{$\circ$}
\DeclareUnicodeCharacter{1D4E5}{$\mathcal{V}$}
\DeclareUnicodeCharacter{1D4D8}{$\mathcal{I}$}
\DeclareUnicodeCharacter{1D4D0}{$\mathcal{A}$}
\DeclareUnicodeCharacter{1D4D6}{$\mathcal{G}$}
\DeclareUnicodeCharacter{211D}{$\mathbb{R}$}

% --------------------------------------------------------------------