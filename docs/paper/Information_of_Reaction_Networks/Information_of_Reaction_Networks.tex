\documentclass[conference]{IEEEtran}
\IEEEoverridecommandlockouts
% The preceding line is only needed to identify funding in the first footnote. If that is unneeded, please comment it out.
\usepackage{cite}
\usepackage{amsmath,amssymb,amsfonts}
\usepackage{algorithmic}
\usepackage{graphicx}
\usepackage{textcomp}
\usepackage{xcolor}
\def\BibTeX{{\rm B\kern-.05em{\sc i\kern-.025em b}\kern-.08em
    T\kern-.1667em\lower.7ex\hbox{E}\kern-.125emX}}
\begin{document}

\title{Information of Chemical Reaction Networks
\thanks{Identify applicable funding agency here. If none, delete this.}
}

\author{\IEEEauthorblockN{1\textsuperscript{st} Given Name Surname}
\IEEEauthorblockA{\textit{dept. name of organization (of Aff.)} \\
\textit{name of organization (of Aff.)}\\
City, Country \\
email address or ORCID}
\and
\IEEEauthorblockN{6\textsuperscript{th} Given Name Surname}
\IEEEauthorblockA{\textit{dept. name of organization (of Aff.)} \\
\textit{name of organization (of Aff.)}\\
City, Country \\
email address or ORCID}
}

\maketitle

\begin{abstract}
This document is a model and instructions for \LaTeX.
This and the IEEEtran.cls file define the components of your paper [title, text, heads, etc.]. *CRITICAL: Do Not Use Symbols, Special Characters, Footnotes, 
or Math in Paper Title or Abstract.
\end{abstract}

\begin{IEEEkeywords}
component, formatting, style, styling, insert
\end{IEEEkeywords}

\section{Introduction}

\noindent - Definition of chemical reaction networks\\
- Definition of the (Chemical) Master Equation;\\
- Evaluation of statistical moments of the CME;\\
- Entropy analysis of the solution of the CME;\\
- Thermodynamical analysis of equilibrium and non-equilibrium systems \\

We define a chemical reaction network as a set of $M$ reaction channels such that
%
\begin{equation}
\underline{s}_{\mu,1}\underline{S}_1 + ... + \underline{s}_{\mu,N_r}\underline{S}_{N_r} \overset{k_\mu}{\longrightarrow} \overline{s}_{\mu,1}\overline{S}_1 + ... + \overline{s}_{\mu,N_r}\overline{S}_{N_r}.
\end{equation}
%
The species $S$ acts like a reactant $\underline{S}$ through a channel $\mu = 1,...,M$, producing $\overline{S}$. The stoichiometric coefficients are indicated by $\underline{s}$ and $\overline{s}$ for the reactants and the products. We define a stoichiometric matrix $\nu \in \mathbb{Z}^{M \times N_r}$, such that
%
\begin{equation}
[\nu]_{ij} = \overline{s}_{i,j} - \underline{s}_{i,j}.
\end{equation}
%
We also define $\alpha_{\mu}(x)$ as the probability reaction channel $\mu$ be activated\cite{Gillespie}, to the $\mu$-th reaction to occur. Armed with this we define a chemical master equation as
%
\begin{equation}
\label{eq CME}
\frac{\partial P(x,t | x_0, t_0)}{\partial t} = \sum_{\mu = 1}^M \alpha_\mu(x - \nu_\mu) P(x - \nu_\mu | x_0,t_0) - \alpha_\mu(x) P(x,t|x_0,t_0),
\end{equation}
%
with $\nu_\mu = [\nu]_{\mu,:}$. [Define x].

The chemical master equation is a set of differential equations for the evolution of the transition probability $P(x,t | x_0, t_0)$, between the states $x$ and $x_0$. We are not able to find an analytical solution for $P(x,t | x_0, t_0)$ in general but for special cases[Jahnke]. To obtain solutions numerically, one needs to crop the infinite state space.[Cropping and out-flow]

The system is said {\it closed}, if there is no production or degradation of species. Otherwise the system is said {\it open}. In the closed case the system can only change its state according to $x = x_0 + \xi \nu, \xi \in \mathbb{Z}$. Thus, once defined the initial state $x_0$ of the system, the states in which the system can be are restricted to the transitions that the reactions allow.

The CME operator $A$ (?) is {\it irreducible} if and only if the adjacency matrix of $A$ is strongly connected\cite{Brualdi}. Otherwise, the strongly connected components of $A$ will form a set of the irreducible parts of the system. Each of them has its own unique stationary probability distribution. Thus, the system has a limiting distribution which is a linear combination of those stationary distributions.

A necessary condition for strong connection is that every state to be reachable from every other state, which for closed systems will be always wrong once the system is restricted to a subset of the states. This subset correspond to the states in which the number of species of the systems is conserved. This has another implications as the non-ergodicity of closed systems. The subset of possible states of the system will always be dependent on the initial condition once the later defines the [copy number] of the species. For instance, in reactions like

\begin{equation}
E + A \underset{k_{-1}}{\stackrel{k_1}{\rightleftharpoons}} EA \overset{k_2}{\rightarrow} E + B,
\end{equation}
%
known as Michaelis-Menten enzyme mechanism reaction, a state which has zero copy number of enzyme molecules $E$ can be reached, but the system will never transition from it. This state is said to absorb the probability through the evolution of the Markov chain, and is not strongly connected with the other states. Thus the system is not irreducible.

For open systems, the uniqueness of the stationary distribution is not assured. But for reactions like
%
\begin{equation}
\emptyset \rightarrow A, \quad A \rightarrow \emptyset
\end{equation}
%
called production and degradation, all the states are connected given there will always be a reaction to drive the system the every state. Thus the system is irreducible. The system can be ergodic if all the states are aperiodic [citation needed]. For this, one needs to prove if the states have periodicity equals the unity.
%

\section{Statistics of the solution of the CME}


\begin{thebibliography}{100} % 100 is a random guess of the total number of %references
\bibitem{Gillespie} Gillespie, Daniel T. ``A rigorous derivation of the chemical master equation.'' {\it Physica A: Statistical Mechanics and its Applications}, 188.1-3 (1992): 404-425.

\bibitem{Gillespie2000} Gillespie, Daniel T. ``The chemical Langevin equation.'', {\it The Journal of Chemical Physics} 113.1 (2000): 297-306.

\bibitem{Brualdi} Richard A. Brualdi, Herbert J. Ryser. {\it Combinatorial Matrix Theory}, Cambridge University Press, 1991

\bibitem{vanKampen} van, Kampen N G. {\it Stochastic Processes in Physics and Chemistry}. Elsevier, 2010. 

\bibitem{Kazeev} Kazeev, Vladimir, et al. ``Direct solution of the chemical master equation using quantized tensor trains.'' {\it PLoS computational} biology 10.3 (2014): e1003359.

\bibitem{Ion} Ion, Ion Gabriel, et al. ``Tensor-train approximation of the chemical master equation and its application for parameter inference.'' {\it The Journal of Chemical Physics} 155.3 (2021): 034102.

\bibitem{Dolgov} Dolgov, Sergey, and Boris Khoromskij. ``Simultaneous state‐time approximation of the chemical master equation using tensor product formats.'' {\it Numerical Linear Algebra with Applications} 22.2 (2015): 197-219.

\bibitem{Khammash} Gupta, Ankit, Jan Mikelson, and Mustafa Khammash. ''A finite state projection algorithm for the stationary solution of the chemical master equation.'' {\it The Journal of chemical physics} 147.15 (2017): 154101.

\bibitem{Patrick} Gelß, Patrick. ``The tensor-train format and its applications: Modeling and analysis of chemical reaction networks, catalytic processes, fluid flows, and Brownian dynamics.'' {\it Diss}. 2017.

\bibitem{Udrescu} Udrescu, Tudor. Numerical methods for the chemical master equation. Diss. Verlag nicht ermittelbar, 2012.

\bibitem{Qian2021} Qian, Hong, and Hao Ge. Stochastic Chemical Reaction Systems in Biology. Cham, Switzerland: Springer, 2021.

\bibitem{Esposito} Esposito, Massimiliano. ``Open questions on nonequilibrium thermodynamics of chemical reaction networks.'' {\it Communications Chemistry} 3.1 (2020): 1-3.

\bibitem{QianHong} Qian, Hong, Min Qian, and Xiang Tang. ``Thermodynamics of the general diffusion process: time-reversibility and entropy production.'' {\it Journal of statistical physics} 107.5 (2002): 1129-1141.

\bibitem{QianGe} Ge, Hao, and Hong Qian. ``Physical origins of entropy production, free energy dissipation, and their mathematical representations.'' {\it Physical Review E} 81.5 (2010): 051133.

\bibitem{QianJiang} Jiang, Da-Quan, Min Qian, and Fu-Xi Zhang. ``Entropy production fluctuations of finite Markov chains.'' {\it Journal of Mathematical Physics} 44.9 (2003): 4176-4188.

\bibitem{QianGehao} Ge, Hao, Da-Quan Jiang, and Min Qian. ``Reversibility and entropy production of inhomogeneous Markov chains.'' {\it Journal of applied probability} 43.4 (2006): 1028-1043.

\bibitem{MouLiang} Mou, Chung‐Yuan, and Chung‐Hsien Liang. ``Nonequilibrium thermodynamic analysis of Michaelis–Menten kinetics.'' {\it The Journal of chemical physics} 93.10 (1990): 7314-7320.

\bibitem{ZhangOuyang} Zhang, Dongliang, and Qi Ouyang. ``Nonequilibrium thermodynamics in biochemical systems and its application.'' {\it Entropy} 23.3 (2021): 271.

\end{thebibliography}

\vspace{12pt}

IEEE conference templates contain guidance text for composing and formatting conference papers. Please ensure that all template text is removed from your conference paper prior to submission to the conference. Failure to remove the template text from your paper may result in your paper not being published.

\end{document}
