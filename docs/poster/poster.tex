% Gemini theme
% https://github.com/anishathalye/gemini

\documentclass[final]{beamer}

% ====================
% Packages
% ====================

\usepackage[T1]{fontenc}
\usepackage{caption}
\usepackage{subcaption}
\usepackage[superscript,biblabel]{cite}
\usepackage{lmodern}
\usepackage[scale=1]{beamerposter}
%\usepackage[size=custom,width=90,height=130,scale=1.25]{beamerposter}
\usetheme{gemini}
\usecolortheme{labsix}
\usepackage{graphicx}
\usepackage{svg}
\svgsetup{inkscapeexe=/Applications/Inkscape.app/Contents/MacOS/inkscape}
\DeclareGraphicsRule{.ai}{pdf}{.ai}{}
\usepackage{booktabs}
\usepackage{tikz}
\usetikzlibrary{fit,backgrounds}
\usepackage{pgfplots}
\usepgfplotslibrary{fillbetween}


% ====================
% Lengths
% ====================

% If you have N columns, choose \sepwidth and \colwidth such that
% (N+1)*\sepwidth + N*\colwidth = \paperwidth
\newlength{\sepwidth}
\newlength{\colwidth}
\setlength{\sepwidth}{0.025\paperwidth}
\setlength{\colwidth}{0.3\paperwidth}

\newcommand{\separatorcolumn}{\begin{column}{\sepwidth}\end{column}}

% ====================
% Title
% ====================

\title{Stochastic thermodynamic analysis of the Michaelis-Menten kinetics}

% REMOVED \inst{1}
\author{Filipe P. de Farias\inst{1} \and Francesco Corona\inst{1}  \and Michela Mulas\inst{1}}

\institute[shortinst]{\inst{1}Dept.}

% ====================
% Footer (optional)
% ====================
 %% REMOVED {  %\href{https://www.example.com}{https://www.example.com} 
  %\href{mailto:alyssa.p.hacker@example.com}{alyssa.p.hacker@example.com} }
\footercontent{
  \hfill
  ???
  \hfill
}
% (can be left out to remove footer)

% ====================
% Logo (optional)
% ====================

% use this to include logos on the left and/or right side of the header:
\logoright{\includegraphics[height=10cm,
%page=2,trim={17cm 21cm 3cm 5cm},clip
]{"graphics/brasao2_vertical_monocromatico"}}
\logoleft{\includegraphics[height=10cm]{"example-image-a"}}

% ====================
% Body
% ====================

\begin{document}

\setlength{\abovedisplayskip}{40pt}
\setlength{\belowdisplayskip}{40pt}

\begin{frame}[t]
\begin{columns}[t]
\separatorcolumn

\begin{column}{\colwidth}

\begin{block}{Introduction}
{\bf Stochastic thermodynamics} (ST) deals with the interaction of mesoscopic, nonequilibrium physical systems with heat reservoirs in equilibrium.\cite{peliti2021stochastic} Such interactions are assumed to be the source of the randomness in the dynamics of the system, assigning to it a probability $p_x(t)$ of being in the state $x$ at time $t$.
\begin{itemize}
\item We will use the Michaelis-Menten kinetics as case of study for the ST.
\end{itemize}
\end{block}

\begin{alertblock}{Michaelis-Menten kinetics (MM)}
The system (MM) is composed by a single molecule of enzyme $E$. We assume the enzyme processes a single molecule of substrate $S$ per time. Then the system can be in two states: free enzyme $E$ and complexed $ES$.
\begin{itemize}
\item The reaction network that models the kinetics is:
\begin{equation}
E + S \underset{k_{-1}}{\stackrel{k_1}{\rightleftharpoons}} ES \underset{k_{-2}}{\stackrel{k_2}{\rightleftharpoons}} E + P
\end{equation}
\item The substrate $S$ and the product $P$ are {\bf chemostated}.
\end{itemize}
\end{alertblock}

\begin{figure}
\label{fig 1}
\begin{subfigure}[b]{0.45\textwidth}
\includesvg{graphics/ST-1.svg}
\caption{}
\label{fig 2-state-system}
\end{subfigure}
\hfill
\begin{subfigure}[b]{0.45\textwidth}
\includesvg{graphics/2-state-graph.svg}
\caption{}
\label{fig 2-state-graph}
\end{subfigure}
\caption{In (\subref{fig 2-state-system}) the representation of the MM[Massimiliano REF?] and in (\subref{fig 2-state-graph}) a single realization of the system.}
\end{figure}


The system is kept in contact with a heat bath with temperature $T$.
\begin{itemize}
\item The changes in state of the system are due to energy exchanges with the bath.
\end{itemize}

\begin{block}{Master Equation}
How the probability $p_x(t)$ of the system being in $x \in \{E,ES\}$ change with time, is given by a {\bf master equation}\cite{van2007stochastic}. It reads:
%
\begin{equation}
\frac{dp_x(t)}{dt} = \sum_x W_{x^\prime x} p_x(t) -  W_{x x^\prime}p_{x^\prime}(t) \label{eq CME}
\end{equation}
\begin{itemize}
\item The $W_{x^\prime x}$ is the {\bf probability transition rate} from the state $x^\prime$ to $x$.
\item $W_{x^\prime x}$ forms a {\bf stochastic matrix} $W$ dependent on the kinetics of the chemical reactions.
\item Integrating or sampling \eqref{eq CME} allow us to obtain the probabilities $p_x(t)$.
\end{itemize}
\end{block}

\end{column}

\separatorcolumn

\begin{column}{\colwidth}
\begin{figure}
\label{fig 1}
\begin{subfigure}[b]{0.45\textwidth}
\includesvg{graphics/ST-MM.svg}
\caption{}
\label{fig 2-state-system}
\end{subfigure}
\begin{subfigure}[b]{0.45\textwidth}
\includesvg{graphics/ST-MM-prob.svg}
\caption{}
\label{fig 2-state-system}
\end{subfigure}
\caption{In (\subref{fig 2-state-system}) the representation of the MM[Massimiliano REF?] and in (\subref{fig 2-state-graph}) a single realization of the system.}
\end{figure}

\begin{block}{Stochastic Thermodynamics}
The classical thermodynamics is defined assumed an {\bf equilibrium} situation of the system.
\begin{itemize}
\item In the ST, the equilibrium is held by the bath, the system is allowed to be in nonequilibrium.
%\item When the system reaches the equilibrium, it is also in equilibrium with the bath (Zero-th law).
\item In such case, ST gives that the system has nonnegative average entropy production rate $\dot{S}^{sys}$:
%
\begin{equation}
\frac{\dot{S}^{sys}}{k_B} = \frac{1}{2} \sum_{x \neq x^\prime} \left[ W_{x^\prime x} p_x(t) -  W_{x x^\prime}p_{x^\prime} \right] \ln \frac{p_x(t)}{p_{x^\prime}(t)}.
\end{equation}
%
This expression can be separated in two parts:

\begin{equation}
\frac{\dot{S}^{tot}}{k_B}  = \frac{1}{2} \sum_{x \neq x^\prime} \left[ W_{x^\prime x} p_x(t) -  W_{x x^\prime}p_{x^\prime} \right] \ln \frac{W_{x^\prime x} p_x(t)}{W_{xx^\prime}p_{x^\prime}(t)}
\quad 
\frac{\dot{S}^{bath}}{k_B}  = \frac{1}{2} \sum_{x \neq x^\prime} \left[ W_{x^\prime x} p_x(t) -  W_{x x^\prime}p_{x^\prime} \right] \ln \frac{W_{x^\prime x}}{W_{xx^\prime}}
\end{equation}
The term $\frac{\dot{S}^{bath}}{k_B}$ is the average heat absorbed by the bath when the system jumps between the states, while $\frac{\dot{S}^{tot}}{k_B}$ is the total entropy change (or balance) of both system and bath.

\item If $p_x^{eq}$ is the probability of the system when in equilibrium, ST gives us the {\bf free energy}
%
\begin{equation}
\frac{\dot{F}(t)}{k_B T} = \sum_{x \neq x^\prime} \left[ W_{x^\prime x} p_x(t) -  W_{x x^\prime}p_{x^\prime} \right] \ln \frac{p_x(t)}{p_{x^\prime}^{eq}} = w - T \dot{S}^{tot}.
\end{equation}
The average work $w$ that can be done in the system is then defined in terms of $p_x^{eq}$
\begin{equation}
w(t) = \frac{1}{2} \sum_{x \neq x^\prime} \left[ W_{x^\prime x} p_x(t) -  W_{x x^\prime}p_{x^\prime} \right] \ln \frac{W_{x^\prime x} p_x^{eq}}{W_{xx^\prime}p_{x^\prime}^{eq}}.
\end{equation}
\end{itemize}
\end{block}

\end{column}

\separatorcolumn

\begin{column}{\colwidth}
\begin{block}{References}

\nocite{*}
\footnotesize{\bibliographystyle{plain}\bibliography{poster}}

\end{block}

\end{column}

\separatorcolumn
\end{columns}
\end{frame}

\end{document}
