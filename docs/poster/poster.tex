% Gemini theme
% https://github.com/anishathalye/gemini

\documentclass[final]{beamer}

% ====================
% Packages
% ====================

\usepackage[T1]{fontenc}
\usepackage{caption}
\usepackage{subcaption}
\usepackage[superscript,biblabel]{cite}
\usepackage{lmodern}
\usepackage[scale=1.1]{beamerposter}
%\usepackage[size=custom,width=90,height=130,scale=1.25]{beamerposter}
\usetheme{gemini}
\usecolortheme{labsix}
\usepackage{graphicx}
\usepackage{svg}
\svgsetup{inkscapeexe=/Applications/Inkscape.app/Contents/MacOS/inkscape}
\DeclareGraphicsRule{.ai}{pdf}{.ai}{}
\usepackage{booktabs}
\usepackage{tikz}
\usetikzlibrary{fit,backgrounds}
\usepackage{pgfplots}
\usepgfplotslibrary{fillbetween}


% ====================
% Lengths
% ====================

% If you have N columns, choose \sepwidth and \colwidth such that
% (N+1)*\sepwidth + N*\colwidth = \paperwidth
\newlength{\sepwidth}
\newlength{\colwidth}
\setlength{\sepwidth}{0.025\paperwidth}
\setlength{\colwidth}{0.3\paperwidth}

\newcommand{\separatorcolumn}{\begin{column}{\sepwidth}\end{column}}

% ====================
% Title
% ====================

\title{Stochastic thermodynamic analysis of the Michaelis-Menten kinetics}

% REMOVED \inst{1}
\author{Filipe P. de Farias\inst{1} \and Francesco Corona\inst{1}  \and Michela Mulas\inst{1}}

\institute[shortinst]{\inst{1}Dept.}

% ====================
% Footer (optional)
% ====================
 %% REMOVED {  %\href{https://www.example.com}{https://www.example.com} 
  %\href{mailto:alyssa.p.hacker@example.com}{alyssa.p.hacker@example.com} }
\footercontent{
  \hfill
  ???
  \hfill
}
% (can be left out to remove footer)

% ====================
% Logo (optional)
% ====================

% use this to include logos on the left and/or right side of the header:
\logoright{\includegraphics[height=10cm,
%page=2,trim={17cm 21cm 3cm 5cm},clip
]{"graphics/brasao2_vertical_monocromatico"}}
\logoleft{\includegraphics[height=10cm]{"example-image-a"}}

% ====================
% Body
% ====================

\begin{document}

\setlength{\abovedisplayskip}{40pt}
\setlength{\belowdisplayskip}{40pt}

\begin{frame}[t]
\begin{columns}[t]
\separatorcolumn

\begin{column}{\colwidth}

\begin{block}{Introduction}
{\bf Stochastic thermodynamics} (ST) deals with the interaction of mesoscopic, nonequilibrium physical systems with heat reservoirs in equilibrium.\cite{peliti2021stochastic} Such interactions are assumed to be the source of the randomness in the dynamics of the system, assigning to it a probability $p_x(t)$ of being in the state $x$ at time $t$.
\begin{itemize}
\item We will use the Michaelis-Menten kinetics as case of study for the ST.
\end{itemize}
\end{block}

\begin{alertblock}{Michaelis-Menten kinetics (MM)}
The system (MM) is composed by a single molecule of enzyme $E$. We assume the enzyme processes a single molecule of substrate $S$ per time. Then the system can be in two states: free enzyme $E$ and enzyme-substrate complex $ES$.
\begin{itemize}
\item The reaction network that models the kinetics is:
\begin{equation}
E + S \underset{k_{-1}}{\stackrel{k_1}{\rightleftharpoons}} ES \underset{k_{-2}}{\stackrel{k_2}{\rightleftharpoons}} E + P
\end{equation}
%\item The substrate $S$ and the product $P$ are {\bf chemostated}.
\begin{figure}
\includesvg{graphics/2-state-graph.svg}
\caption{}
\label{fig 2-state-graph}
\end{figure}
\item The observation of a single realization of this system is given in Figure \ref{fig 2-state-graph}. The system fluctuates between the two states until reaches a stationary configuration.
\end{itemize}
\end{alertblock}
%
%\begin{figure}
%\label{fig 1}
%\begin{subfigure}[b]{0.45\textwidth}
%\includesvg{graphics/ST-1.svg}
%\caption{}
%\label{fig 2-state-system}
%\end{subfigure}
%\hfill
%\begin{subfigure}[b]{0.45\textwidth}
%\end{subfigure}
%\caption{In (\subref{fig 2-state-system}) the representation of the MM[Massimiliano REF?] and in (\subref{fig 2-state-graph}) a single realization of the system.}
%\end{figure}
%
%
%The system is kept in contact with a heat bath with temperature $T$.
%\begin{itemize}
%\item The changes in state of the system are due to energy exchanges with the bath.
%\end{itemize}

\begin{block}{Master Equation}
The probability $p_x(t)$ of the system being in $x \in \{E,ES\}$ and how it changes with time, is given by a {\bf master equation}\cite{van2007stochastic}. It reads:
%
\begin{equation}
\frac{dp_x(t)}{dt} = \sum_x W_{x^\prime x} p_x(t) -  W_{x x^\prime}p_{x^\prime}(t) \label{eq CME}
\end{equation}
\begin{itemize}
\item The $W_{x^\prime x}$ is the {\bf probability transition rate} from the state $x^\prime$ to $x$, it forms a {\bf stochastic matrix} $W$ dependent on the kinetics of the chemical reactions\cite{GILLESPIE1976403}:
\begin{equation}
W_{x^\prime x} = \sum_\nu \prod_i k_\nu \frac{x_{i,\nu}!}{(x_{i,\nu} - s_{i,\nu})!}
\end{equation}
$x_{i,\nu} = $ \# of molecules in the system of the i-th reactant in the $\nu$-th reaction.\\
$s_{i,\nu} = $ \# of molecules of the i-th reactant participating in the $\nu$-th reaction.
%\item Integrating or sampling \eqref{eq CME} allow us to obtain the probabilities $p_x(t)$.
%\item 
\end{itemize}
\end{block}

\end{column}

\separatorcolumn

\begin{column}{\colwidth}

\begin{block}{Stochastic Thermodynamics}
\vskip10pt
\begin{figure}
\label{fig 1}
\begin{subfigure}[b]{0.45\textwidth}
\centering
\includesvg{graphics/ST-1.svg}
\caption{}
\label{fig 2-state-system}
\end{subfigure}
\begin{subfigure}[b]{0.45\textwidth}
\centering
\includesvg[scale=1.1]{graphics/ST-MM-prob.svg}
\caption{}
\label{fig 2-prob-evol}
\end{subfigure}
\caption{In (\subref{fig 2-state-system}) the representation of MM\cite{esposito2023} and in (\subref{fig 2-prob-evol}) the evolution of the probability for each state.}
\end{figure}


The classical thermodynamics is defined assumed an {\bf equilibrium} situation of the system.
\begin{itemize}
\item In the ST, the equilibrium is held by the bath, the system is allowed to be in {\bf nonequilibrium}.
%\item When the system reaches the equilibrium, it is also in equilibrium with the bath (Zero-th law).
\item In such case, ST gives that the system has nonnegative {\bf average entropy production rate} $\dot{S}^{sys}$:
%
\begin{equation}
\dot{S}^{sys} = k_B T \frac{1}{2} \sum_{x \neq x^\prime} \left[ W_{x^\prime x} p_x(t) -  W_{x x^\prime}p_{x^\prime} \right] \ln \frac{p_x(t)}{p_{x^\prime}(t)}.
\end{equation}
%
This expression can be separated in two parts:
\begin{subequations}
\begin{equation}
\dot{S}^{tot}  = k_B T\frac{1}{2} \sum_{x \neq x^\prime} \left[ W_{x^\prime x} p_x(t) -  W_{x x^\prime}p_{x^\prime} \right] \ln \frac{W_{x^\prime x} p_x(t)}{W_{xx^\prime}p_{x^\prime}(t)}
\end{equation}
\begin{equation}
\dot{S}^{bath}  = k_B T \frac{1}{2} \sum_{x \neq x^\prime} \left[ W_{x^\prime x} p_x(t) -  W_{x x^\prime}p_{x^\prime}(t) \right] \ln \frac{W_{x^\prime x}}{W_{xx^\prime}}
\end{equation}
\end{subequations}
The term $\dot{S}^{bath}$ is the average heat absorbed by the bath when the system jumps between the states, while $\dot{S}^{tot}$ is the total entropy change (or balance) of the universe (system plus bath).

\item If $p_x^{eq}$ is the probability of the system when in equilibrium, ST gives us the {\bf generalized free energy} rate\cite{Qian_2021}
%
\begin{equation}
\label{eq noneq free en}
\dot{F}(t) - \dot{F}^{eq}(t)= \sum_{x \neq x^\prime} \left[ W_{x^\prime x} p_x(t) -  W_{x x^\prime}p_{x^\prime} \right] \ln \frac{p_x(t)}{p_{x^\prime}^{eq}},
\end{equation}
%
where $\dot{F}^{eq}$ is the equilibrium free energy.

\item The nonequilibrium free energy rate in \eqref{eq noneq free en} is defined as the information $I$ needed to specify the nonequilibrium state\cite{Esposito_2011}, thus
%
\begin{equation}
{F}(t) - {F}^{eq}(t) = TI(t) \equiv TD_{KL}(p_x(t) || p_x^{eq}(t)) \geq 0
\end{equation}
%
where $D_{KL}$ is the Kullback-Leibler divergence measuring the ``difference'' between the nonequilibrium and the equilibrium probability distributions.
%The difference between  the rate of work which is done when we manipulate the system $\dot{w}$ and the one available to be extracted $\dot{F}^{eq}$ is
%\begin{equation}
%\dot{w} - \dot{F}^{eq}= \frac{1}{2} \sum_{x \neq x^\prime} \left[ W_{x^\prime x} p_x(t) - W_{x x^\prime}p_{x^\prime} \right] \ln \frac{W_{x^\prime x} p_x^{eq}}{W_{xx^\prime}p_{x^\prime}^{eq}}.
%\end{equation}
%\item The thermodynamic flux and force are defined as, respectively
%%
%\begin{equation}
%J_{xx^\prime} = W_{x^\prime x} p_x(t) -  W_{x x^\prime}p_{x^\prime}(t) \qquad A_{xx^\prime} =  \ln \frac{W_{x^\prime x} p_x(t)}{W_{xx^\prime}p_{x^\prime}(t)}
%\label{eq therm-flux-force}
%\end{equation}
\end{itemize}
\end{block}

\end{column}

\separatorcolumn

\begin{column}{\colwidth}

\begin{block}{Analysis}
\begin{figure}
\vskip-30pt 
\begin{center}
\includesvg[scale=1.25]{graphics/ST-MM-2.svg}
\end{center}
\label{fig 2-state-system}
\caption{On top the evaluations of equations (4) to (6) in units of $k_B T$, to the MM system with $k_{1} = 0.5$, $k_{-1} = 0.005$, $k_{2} = 0.1$ and $k_{-2} = 0.0$. On the bottom, the time evolution of the probability of the system $p_{x}$ in each state and the respective equilibrium probability $p_{x}^{eq}$.}
\end{figure}

The system reaches the steady-state, which for the case of study happens to be also the equilibrium in about 10 seconds:
%
\begin{itemize}
\item Both\cite{Qian_2021} the thermodynamic force and flux vanish in the steady-state, which is also an equilibrium.

%\item It can be verified that the system obeys {\bf detailed balance}, which connects the math and the physics by recovering the Boltzmann distribution.

\item It can be noticed also that all the free energy is used to produce entropy.

\item This could be different if one realize work on the system, and it's subject of further research.
\end{itemize}
\end{block}
\begin{block}{References}

\nocite{*}
\footnotesize{\bibliographystyle{plain}\bibliography{poster}}

\end{block}

\end{column}

\separatorcolumn
\end{columns}
\end{frame}

\end{document}
