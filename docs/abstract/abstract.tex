\documentclass[11pt]{amsart}
\usepackage{geometry}                % See geometry.pdf to learn the layout options. There are lots.
\geometry{letterpaper}                   % ... or a4paper or a5paper or ... 
%\geometry{landscape}                % Activate for for rotated page geometry
%\usepackage[parfill]{parskip}    % Activate to begin paragraphs with an empty line rather than an indent
\usepackage{graphicx}
\usepackage{amssymb}
\usepackage{epstopdf}
\DeclareGraphicsRule{.tif}{png}{.png}{`convert #1 `dirname #1`/`basename #1 .tif`.png}
\usepackage[superscript,biblabel]{cite}

\title{Mesoscopic thermodynamics of single-particle enzymatic reactions}
\author{Filipe P. de Farias, Francesco Corona, Michela Mulas}
%\date{}                                           % Activate to display a given date or no date

\begin{document}
\maketitle

In this model, we consider a Michaelis-Menten chemical reaction network with a reversible product formation. The system consists of a single particle of an enzyme that can exist in two states: either as free enzyme or as an enzyme complexed with the substrate. Both the substrate and product are maintained at constant concentrations. The system is in contact with a larger thermal bath at equilibrium, ensuring a constant temperature throughout. The weak coupling between the system and the bath ensures that any exchange of quantities between them does not disturb the equilibrium of the bath.

To study this system, we employ two complementary approaches: kinetic Monte Carlo methods and ordinary differential equation solutions for the chemical master equation. Kinetic Monte Carlo methods enable us to simulate the behavior of the chemical system and obtain realizations that provide insights into its dynamics. On the other hand, ordinary differential equation solutions allow us to solve the chemical master equation and obtain the time evolution of probabilities for each state of the system.

To compute thermodynamic quantities such as heat, entropy, and free energy, we utilize the stochastic thermodynamics framework. These quantities are exchanged with the thermal bath, and their calculation relies on an information-theoretic approach to entropy. By considering the probability distribution of the system, we obtain a stochastic entropy that is proportional to the Shannon entropy and depends on the bath temperature and Boltzmann constant. Averaging over the states of the system and differentiating with respect to time, we can determine the entropy balance rate, which represents the entropy produced internally by the system and the entropy exchanged with the bath. Additionally, the free energy is computed as proportional to the Kullback-Leibler divergence between the system's probability distribution and its equilibrium distribution.

In summary, this model incorporates the Michaelis-Menten chemical reaction network with reversible product formation. We utilize kinetic Monte Carlo methods and ordinary differential equation solutions to examine the system's behavior, while employing the stochastic thermodynamics framework to compute thermodynamic quantities such as heat, entropy, and free energy. These approaches provide valuable insights into the system's dynamics and thermodynamic properties.

\nocite{*}

\footnotesize{\bibliographystyle{plain}\bibliography{abstract}}
\end{document}  